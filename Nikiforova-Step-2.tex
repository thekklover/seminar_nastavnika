\documentclass[12pt]{article}
\usepackage[utf8]{inputenc}
\usepackage[T1]{fontenc}
\usepackage{amsmath,amsfonts,amssymb}
\usepackage{graphicx}
\usepackage{a4wide}\title{Industrial project description (Mortgage credit scoring)}
%\author{not specified}
\date{}
\begin{document}
\maketitle

%\begin{abstract}
 Role: Expert.
%\end{abstract}
% \paragraph{Keywords:} The Art On Scientific Research, Abstract Reconstruction, Please Put Yours 


\section{Planning the industrial research project}
Before planning the research, the analyst and (\textbf{expert}) discuss the key issues. After the long dash~--- our remarks.

\begin{enumerate}
\item Goal of the project: The objective is to develop an accurate credit scoring model that predicts a borrower’s likelihood of default, improving lending decisions and risk management.
\item Applied problem solved in the project:This project tackles the issue of elevated credit risk in mortgage. The scoring model will minimize loan defaults and optimize approval processes. The results will be used in leading tailored loan products and improved risk management strategies. The outcomes will be demonstrated through key business metrics, such as the reduction of expected losses, with operational efficiency improvements highlighted in visual dashboards and detailed impact reports.
\item Description of historical measured data: Dataset should include mortgage performance metrics, homeowner profiles, payment histories and applications for at least a decade.
\item Quality criteria: Quality will be assessed via metrics such as ROC-AUC, accuracy, precision, and recall. The report will include model performance summaries and error analyses.
\item Project feasibility: Feasibility will be demonstrated through exploratory data analysis, identifying potential risks such as data bias, model overfitting and integration challenges with existing systems. A failure analysis plan will identify and mitigate risks.
\item Conditions necessary for successful project implementation: success requires access to a comprehensive clean and structured dataset and the collaboration of cross-functional teams.
\item Solution methods: We will use machine learning libraries (e.g. Scikit-learn, TensorFlow) and test different models, such as logistic regression and decision trees, to find the optimal scoring model.
\end{enumerate}

\section{Research or development?}
The credit scoring model is expected to be used for an initial period of one year, during which time it will be evaluated and updated on a regular basis. As market conditions and consumer behaviour change, the model may need to be recalibrated or retrained with new data to maintain its accuracy and relevance. This proactive approach will ensure that the model maintains its competitive advantage in credit scoring.

In the future, we expect that advances in machine learning techniques and automation will lead to the development of more sophisticated models, potentially incorporating alternative data sources and real-time analytics. These next-generation models may replace the current credit scoring system, improving the predictive power and decision-making processes in credit assessments.

%\bibliographystyle{unsrt}
%\bibliography{Name-theArt}
\end{document}